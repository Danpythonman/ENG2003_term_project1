\section{Machine Learning's Use in Cybersecurity}
% Cybersecurity systems include many different systems, like firewalls and antivirus software.
% Once such type is called an intrusion detection system (IDS).
% IDSs help to differentiate between authorized and unauthorized uses of a provided service \cite{xin2018}.
% This is a technology that machine learning has the most potential to impact.

Traditionally, cybersecurity algorithms were written manually from heuristics \cite{sarker_kayes_badsha_2020}.
But the rapid growth of the internet and technology in general has led to constantly changing cybersecurity threats.
As a result, these manually written heuristic algorithms are insufficient --- they cannot keep up with the evolving threats \cite{sarker_kayes_badsha_2020}.
Machine learning offers a solution to this problem.

Machine learning models are able to ``learn'' certain data patterns to predict behavior \cite{sarker_kayes_badsha_2020}.
In this case, their goal is to predict whether some online activity is malicious or legitimate.
To accomplish this, the model must be trained with training data and tested to ensure it is effective.
This first involves data-driven tasks, like gathering and cleaning data \cite{sarker_kayes_badsha_2020}.
This data can then be used to train the model, which may take seconds to days, depending on the algorithm chosen \cite{xin2018}.
Once the model is trained, it must be tested to ensure it is accurately detecting malicious activity, which also takes a variable amount of time depending on the machine learning algorithm chosen \cite{xin2018}.
Figure 1 shows a diagram of this process.

\begin{figure}[H]
    \centering
    \includegraphics[width=0.7\textwidth]{echosec_machine_learning_diagram.png}
    \caption{A Diagram of a Typical Machine Learning Process. \cite{echosec}}
\end{figure}

\section{Evidence of Machine Learning's Effectiveness}
Machine learning algorithms are already being employed to detect cyberattacks.
The following are a few case studies of machine learning being successfully implemented.

\subsection{Windows Defender Antivirus}
In 2018, a new malware attack campaign was launched against over a thousand users of Windows 7 Pro \cite{microsoft2018}.
The Windows Defender Antivirus features lightweight machine learning models built into the client, which responded immediately to the attack \cite{microsoft2018}.
These models detected a high probability of maliciousness, so they sent data to the Windows Defender Antivirus cloud protection service, which runs more complex machine learning models \cite{microsoft2018}.
Through this, the cloud protection service correctly identified the requests as a cyberattack and responded back to the clients, instructing them to block the attack \cite{microsoft2018}.
The use of machine learning algorithms were able to protect thousands of users from a cyberattack with no human intervention.

\section{Drawbacks of Machine Learning for Cybersecurity}
In its current state, machine learning has many drawbacks for use in cybersecurity, some of which make it infeasible for organizations to use.

\subsection{Availability of Datasets}
Since cyberattacks can be complex and varied, large datasets are needed to train the machine learning models to ensure they can protect against all attacks.
This is not the case with the existing datasets available.
Current datasets contain lots of old data and redundant information \cite{xin2018}.
This can be somewhat improved after cleaning the data, but even then there is the issue of volume --- there is not enough data to properly train the models \cite{xin2018}.
As a result, the machine learning models are not totally equipped for identifying new cyberattacks.
This also introduces a barrier of entry, as larger organizations may be able to work around these issues, but smaller organizations do not have the resources to do so.

\subsection{Lack of Research and Adoption}
While the field of machine learning receives lots of research, this research is mainly focused on deep-learning algorithms for applications like self-driving cars \cite{grandchallenge2019}.
Machine learning for cybersecurity purposes has yet to receive this same amount of attention.
Due to this lack of research, widely adopted machine learning models for cybersecurity are limited, using mostly rule-based techniques \cite{grandchallenge2019}.
Furthermore, this lack of research introduces inconsistency across organizations \cite{grandchallenge2019}.
To be most effective, cybersecurity models need to have consistent behavior for any attack that may occur.
This requires cooperation and research to keep all parts of the internet secure.

\section{Discussion}
