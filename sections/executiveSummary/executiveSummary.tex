\pagenumbering{roman}

\section*{Executive Summary}
\addcontentsline{toc}{section}{Executive Summary}
\setcounter{page}{2}
The internet is one of the most complex systems ever engineered.
Utilizing this complexity, cybercriminals have an abundance of opportunities to wreak havoc through the internet.
The options for attacks range from sending fake emails in an attempt to steal someone's password, to shutting down entire power grids.
As industries become more dependent on technology and the internet, the latter case is more feasible for cybercriminals.

The engineering grand challenge of securing cyberspace addresses this problem.
The challenge acknowledges that the massive scope of this problem, and that it will only get worse as technology dependence increases.
This challenge falls into the cross-cutting theme of security, which needs to be prioritized for these reasons.
Creating a secure cyberspace is necessary for the proper functioning of industry and prevention of costly damages.

An emerging technology that addresses this grand challenge is machine learning.
Machine learning models can be trained using training data to detect patterns in new data.
This is especially useful for this challenge, as current pattern recognition software for cybersecurity is programmed manually and is unable to keep up with constantly changing cyberattacks.
Machine learning has an opportunity here to provide automatic detection of malicious activity that can evolve along with the evolving threats.

Initial tests and implementations of this technology are promising.
MalDozer, a machine learning malware detection system for the Android operating system, has shown in experiments to be about 96\% accurate with a false positive rate of about 2\%.
Outside of experiments, the Windows Defender Antivirus, an antivirus software developed by Microsoft, has already implemented machine learning in its production software.
It has faced a cyberattack targeting over one thousand Windows users, to which it correctly identified malicious activity and blocked the attack.

Although promising, the machine learning's use for cybersecurity currently has some drawbacks.
One drawback is that there is not enough datasets to train the machine learning models with.
If not properly trained, the models will not be effective in detecting malicious activity.
Also, while machine learning receives much research, most of it is not directed to cybersecurity.
As a result, the technology still needs to be researched before it can be widely adopted.

Despite these drawbacks, machine learning is a promising solution for the grand challenge.
With more research and resources, this technology will be capable of solving much of the challenge and secure a large part of cyberspace.
