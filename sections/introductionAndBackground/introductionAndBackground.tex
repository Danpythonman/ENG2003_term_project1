\section{Introduction and Background}
Technology and the internet have been growing more and more pervasive over the last few decades, and this shows no sign of stopping.
This comes with many benefits, like an abundance of readily available information, almost instantaneous communication, and the entire industry of online markets.
However, this also comes with consequences.
Perhaps one of the most pressing issues with the growth of technology is the introduction of a new type of malicious activity --- cybercrime.

Cybercrime is the use of the Internet for illegal purposes.
This includes crimes targeting individuals, such as malware, ransomware, phishing, and identity theft.
But with the increasing dependance of industry on the internet, this cybercrime includes crimes that target populations and large systems.
For example, cyberattacks can target power grids, military systems.

The engineering grand challenge of securing cyberspace seeks to address this problem.
The internet is one of the most complex systems ever engineered.

Such cyberattacks have happened.
On June 1, 2020, the University of California, San Francisco, was hacked by a ransomware campaign that threatened to release confidential information, to which the university paid approximately \$1.14 million to the group \cite{winder2020}.
In fact, in 2015, cybercrime was estimated to have had a global cost of \$3 trillion \cite{microsoft2016}, and is expected predicted to reach \$10.5 trillion by 2025 \cite{morgan2020}.

As is evident by successful cyberattacks of reputable organizations, current methods of dealing with these attacks are insufficient.
The main method of developing cybersecurity systems is finding fixes for vulnerabilities only when they are unfortunately discovered.
