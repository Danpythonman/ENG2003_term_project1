\section{Conclusion}
The engineering grand challenge of securing cyberspace has a level of complexity and difficulty that is daunting.
However, this complexity and difficulty can be matched with machine learning.

With the ability to find patterns in data, machine learning shows potential as a detection system for malicious activity.
Even though the cyber threats it must respond to are constantly evolving, machine learning models will remain effective because they can evolve with the threats.
This has been shown both in experimental settings and in real cyberattacks.

In its early stages, this technology is proving to be accurate and effective, and this will only improve with research.
As barriers to entry fade and more research and datasets are compiled, machine learning will be imperative for securing cyberspace.
